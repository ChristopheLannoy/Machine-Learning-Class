\documentclass{article}

\usepackage[utf8]{inputenc}
\usepackage[T1]{fontenc}
%\usepackage[latin1]{inputenc}
%\usepackage[utf8]{fontenc}
\usepackage{amsmath}
\usepackage{amsfonts}
\usepackage{amssymb}
\usepackage{pifont}
\setlength{\parskip}{2pt}
\newcommand{\argmax}{\operatornamewithlimits{argmax}}

\author{Emmanuel Rachelson}
\title{Pre-class activities\\Naive Bayes Classification}
\date{}

\begin{document}

\maketitle

\section{Cluedo}

British people write ``rigour'' while US citizens write ``rigor''. A suspect in a hotel wrote this word on a piece of paper, then burnt it and left it behind to be found by the police. Only one letter remains readable; it is a vowel. It is known that 40\% of the hotel's clients are British and 60\% are Americans.\\
\noindent \ding{43} What is the probability the suspect is British?\\

{\it By using Bayes formula, one obtains that the probability of the suspect being British is equal to the likelihood of finding a vowel in the word provided that a British citizen wrote it, times the probability of meeting a British citizen in this hotel, divided by the probability of finding a vowel in this word in general.
\begin{align*}
\mathbb{P}(British | vowel) & = \frac{\mathbb{P}(vowel | British) \mathbb{P}(British)}{\mathbb{P}(vowel)}\\
& = \frac{\mathbb{P}(vowel | British) \mathbb{P}(British)}{\mathbb{P}(vowel | British) \mathbb{P}(British) + \mathbb{P}(vowel | US) \mathbb{P}(US)}\\
& = \frac{ \frac{3}{6} \frac{40}{100} }{ \frac{3}{6} \frac{40}{100} + \frac{2}{5} \frac{60}{100} }\\
& = \frac{5}{11}
\end{align*}
}

\noindent \ding{43} Suppose now there are more than two english-speaking nationalities staying in this hotel. What part of the previous calculation is unecessary to establish the most probable citizenship of the suspect?

{\it To establish the most probable citizenship of the suspect, one needs to compare $\mathbb{P}(country | vowel)$ for all values of $country$ and pick the one with the highest estimated probability. Between all these calculations, only the numerator changes, hence:
\begin{equation*}
\text{most probable citizenship} = \argmax_{country} \mathbb{P}(vowel | country) \mathbb{P}(country)
\end{equation*}
}

\section{[Optional] Quality check}

In a manufacturing line, 1\% of the production has a defect. A performance test allows to filter out 95\% of faulty products but also excludes 2\% of acceptable ones.\\
\noindent \ding{43} What is the probability of a control error?\\

{\it The probability of a control error is the probability of either a false positive (FP) or a false negative (FN). Let us consider the events ``faulty product'' (F) and ``rejected product'' (R). The data tells us that:
\begin{align*}
\mathbb{P}(F) & = 0.01\\
\mathbb{P}(R|F) & = 0.95\\
\mathbb{P}(R|\bar{F}) = 0.02\\
\end{align*}
Then:
\begin{align*}
\mathbb{P}(FN) & = \mathbb{P}(F\wedge\bar{R}) = \mathbb{P}(\bar{R}|F)\mathbb{P}(F) = \left(1-\mathbb{P}\left(R|F\right)\right)\mathbb{P}(F)\\
\mathbb{P}(FP) & = \mathbb{P}(\bar{F}\wedge R) = \mathbb{P}(R | \bar{F})\mathbb{P}(\bar{F})
\end{align*}
Thus:
\begin{equation*}
\mathbb{P}(FN \lor FP) = \left(1-0.95\right)0.01 + 0.02 \left(1-0.01 \right) = 0.0005 + 0.0198 = 0.0203
\end{equation*}
Interestingly, the mot probable control errors concern acceptable products rejected by mistake.
}

\noindent \ding{43} What is the probability of a letting a faulty product go through?

{\it This probability is $\mathbb{P}(F|\bar{R})$. By application of Bayes' theorem:

\begin{align*}
\mathbb{P}(F|\bar{R}) &= \frac{\mathbb{P}(\bar{R}|F)\mathbb{P}(F)}{\mathbb{P}(\bar{R})}\\
&= \frac{\left(1-\mathbb{P}(R|F)\right)\mathbb{P}(F)}{ \mathbb{P}(\bar{R}|F)\mathbb{P}(F) + \mathbb{P}(\bar{R}|\bar{F})\mathbb{P}(\bar{F}) }\\
&= \frac{\left(1-\mathbb{P}(R|F)\right)\mathbb{P}(F)}{ \left(1-\mathbb{P}(R|F)\right)\mathbb{P}(F) + \left(1-\mathbb{P}(R|\bar{F})\right)\left(1-\mathbb{P}(F)\right) }\\
&= \frac{ (1-0.95)0.01 }{ (1-0.95)0.01 + (1-0.02)(1-0.01) }\\
&= 0,000515092
\end{align*}

\end{document}
