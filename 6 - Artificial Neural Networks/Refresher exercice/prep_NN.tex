\documentclass{article}

\usepackage[utf8]{inputenc}
\usepackage[T1]{fontenc}
%\usepackage[latin1]{inputenc}
%\usepackage[utf8]{fontenc}
\usepackage{amsmath}
\usepackage{amsfonts}
\usepackage{amssymb}
\usepackage{pifont}
\usepackage{graphicx}
\setlength{\parskip}{2pt}

\author{Emmanuel Rachelson}
\title{Pre-class activities\\Artifical Neural Networks}
\date{}

\newcommand{\R}{\ensuremath{\mathbb{R}}}

\begin{document}

\maketitle

\section{Derivation: the chain rule and total derivatives}

\begin{enumerate}
	\item Consider a function $f:\R^p\rightarrow \R^q$. Recall the expression of this function's Jacobian matrix.
	\item The total derivative $Df_{\hat{x}}(h)$ of $f$ in $\hat{x}$ is a linear operator. Recall its definition (for the same $f:\R^p\rightarrow \R^q$ function as above).
	\item Consider two functions, $g:\R\rightarrow\R^p$ and $f:\R^p\rightarrow\R$. Let $F=f\circ g$ be the composite function such that $F(x)=f(g(x))$. Write the derivative of $F$ with respect to $x$ as an expression of the partial derivatives of $f$ and $g$.
	\item Now suppose that in the example above, all the $g_k$ functions are identity functions, that is $g(x) = [x,\ldots, x]^T$. How does the total derivative of $F$ simplify?
	\item Finally, consider two functions $g:\R\rightarrow \R^p$ and $f:\R^p\rightarrow \R^q$. As previously, lets $F=f\circ g$ be the composite function. Write the total derivative of $F$ as an expression of the partial derivatives of $f$ and $g$.
\end{enumerate}

\end{document}
