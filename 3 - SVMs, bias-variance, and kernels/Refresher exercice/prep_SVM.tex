\documentclass{article}

\usepackage[utf8]{inputenc}
\usepackage[T1]{fontenc}
%\usepackage[latin1]{inputenc}
%\usepackage[utf8]{fontenc}
\usepackage{amsmath}
\usepackage{amsfonts}
\usepackage{amssymb}
\usepackage{pifont}
\usepackage{graphicx}
\setlength{\parskip}{2pt}

\author{Emmanuel Rachelson}
\title{Pre-class activities\\Support Vector Machines and Kernels}
\date{}

\begin{document}

\maketitle

\section{Quadratic programming}

%% Consider the (very academic!) optimization problem:
%% \begin{gather*}
%% \min_{x\in \mathbb{R}^n} \frac{1}{2}\|x\|^2\\
%% s.t. \ \forall j=1...N, \quad a_j^T x \leq c_j
%% \end{gather*}
%% You should recognize a quadratic programming problem here.\\
%% \noindent \ding{43} Write down the problem's Lagrangian.\\
%% \noindent \ding{43} Why are the constraints qualified?\\
%% \noindent \ding{43} Write Karush-Kuhn-Tucker's first order conditions. Recall what they mean.

Suppose a collection of $N$ pairs $\{(x_i,y_i)\}_{i=1..N}$ with $x_i\in \mathbb{R}^n$ and $y_i \in \{-1;1\}$ and consider the following optimization problem:
\begin{gather*}
\min_{w\in\mathbb{R}^n} \frac{1}{2}\|w\|^2\\
s.t. \ \forall j=1...N, \quad y_i \left( w^T x_i \right) \leq 1
\end{gather*}

\noindent \ding{43} Recall the name of this type of optimization problem.

\noindent \ding{43} Write down the problem's Lagrangian.

\noindent \ding{43} Why are the constraints qualified?

\noindent \ding{43} Write Karush-Kuhn-Tucker's first order conditions. Recall what they mean.

\noindent \ding{43} Recall the duality theorem in Differentiable Optimization and write the dual form of the above optimization problem.

\section{Insensitive Least squares regression}

Suppose a collection of $N$ points $\{(x_i,y_i)\}_{i=1..N}$ with $x_i\in \mathbb{R}^n$ and $y_i \in \mathbb{R}$ and consider the associated regression problem. Suppose also that a family of functions $\Phi=\{\phi_j\}_{j=1..p}, \ \phi_j:\mathbb{R}^n \rightarrow \mathbb{R}$ is provided and that the solution to the regression problem should lie in the space spanned by $\Phi$ (that is, the regression function $f$ should have the form $f=\sum_{j=1}^p w_j \phi_j$).

\noindent \ding{43} Write the regression problem as a least squares minimization problem.

Consider the data plotted in Figure \ref{fig:3}. 
\begin{figure}[h!]
\begin{center}
\includegraphics[width=8cm]{img/example4.pdf}
\end{center}
\caption{Regression data}
\label{fig:3}
\end{figure}

Apart from the outlier at $x=1.05$, the data fits the $y=x$ relation perfectly.
%However, a least squares regression will not find the $y=x$ relation because of this outlier point.
In this case, advanced feature functions do not seem necessary, so the $j$th feature is actually the $j$th component of $x$. Thus the problem is a simple $y=w^Tx$ regression problem.

\noindent \ding{43} Can you think of a formulation of the regression problem that would be robust to noise? For example, that would discard any noisy point that stays inside a ``tube'' of width $\epsilon$ around the inferred function?

\section{The trick of the additional dimension}

Consider the following test data where $x$ is a voltage measurement and $y$ indicates whether an electronic component failed under that voltage:

\begin{tabular}{|c|c|c|c|c|c|c|c|c|c|}
\hline
$x$ & 0.3 & 0.7 & 1.1 & 1.8 & 2.5 & 3.0 & 3.3 & 3.5 & 3.7 \\
\hline
$y$ & -1  & -1  &  -1 &  1  &  1  &  1  & 1   & -1  & -1\\
\hline
\end{tabular}

\noindent \ding{43} Figure \ref{fig:1} shows a graphical display of the above data set. One wishes to linearly separate the data. What is the (very naive) general form of a linear classifier on this data ? What is the best training error one can obtain with such a classifier ? 

\begin{figure}[h!]
\begin{center}
\includegraphics[width=8cm]{img/example1.pdf}
\end{center}
\label{fig:1}
\caption{Raw measurements}
\end{figure}

A smart engineer decides to plot the same data but enriches the description by adding a second axis representing $(2-x)^2$. The data set becomes:

\begin{tabular}{|c|c|c|c|c|c|c|c|c|c|}
\hline
$z_1=x$ & 0.3 & 0.7 & 1.1 & 1.8 & 2.5 & 3.0 & 3.3 & 3.5 & 3.7 \\
\hline
$z_2=(2-x)^2$ & 2.89 & 1.69 & 0.81 & 0.04 & 0.25 & 1 & 1.69 & 2.25 & 2.89\\
\hline
$y$ & -1  & -1  &  -1 &  1  &  1  &  1  & 1   & -1  & -1\\
\hline
\end{tabular}
The new graphical representation is displayed on Figure \ref{fig:2}.
\begin{figure}[h!]
\begin{center}
\includegraphics[width=8cm]{img/example2.pdf}
\end{center}
\label{fig:2}
\caption{Enriched representation}
\end{figure}

\noindent \ding{43} Did that operation seem to help the linear classification task? What lessons can one draw?

\end{document}
